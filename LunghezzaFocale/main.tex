\documentclass{article}
\usepackage[italian]{babel}
\usepackage[tmargin=2cm,rmargin=1.5in,lmargin=1.5in,margin=0.85in,bmargin=2cm,footskip=.2in]{geometry}
\usepackage{siunitx}
\sisetup{separate-uncertainty=true, per-mode=fraction, parse-numbers=true}
\usepackage{caption}
\usepackage[T1]{fontenc}
\usepackage{bookmark}
\usepackage{mathcomp}
\usepackage{graphicx}
\usepackage{multicol}
\usepackage{booktabs}
\usepackage{amsmath,amsfonts,amsthm,amssymb,mathtools}
\hypersetup{
	pdftitle={Relazione pendolo fisico},
	colorlinks=true, linkcolor=doc!90,
	bookmarksnumbered=true,
	bookmarksopen=true
}
\usepackage{blindtext}
\usepackage{wrapfig}
\usepackage{listings}
\usepackage{xcolor}
\usepackage{float}
\usepackage{tikz}
\usepackage{multirow}
\usepackage{biblatex}
\definecolor{codegreen}{rgb}{0,0.6,0}
\definecolor{codegray}{rgb}{0.5,0.5,0.5}
\definecolor{codepurple}{rgb}{0.58,0,0.82}
\definecolor{backcolour}{rgb}{0.95,0.95,0.92}
\definecolor{doc}{rgb}{0,0,0}
\lstdefinestyle{code}{
    backgroundcolor=\color{backcolour},   
    commentstyle=\color{codegreen},
    keywordstyle=\color{magenta},
    numberstyle=\tiny\color{codegray},
    stringstyle=\color{codepurple},
    basicstyle=\ttfamily\footnotesize,
    breakatwhitespace=false,         
    breaklines=true,                 
    captionpos=b,                    
    keepspaces=true,                                     
    showspaces=false,                
    showstringspaces=false,
    showtabs=false,                  
    tabsize=2,
    inputencoding=ansinew,
    extendedchars=true,
    numbers=left,                    
    numbersep=5pt
}

\lstset{style=code}
\usepackage[varbb]{newpxmath}
\usepackage{circuitikz}
\captionsetup{labelfont={bf, sc}}
\title{Lunghezza Focale}
\author{Francesco Sermi}
\date{\today}

\begin{document}
	\maketitle
	\section{Scopo}
	Misurare la lunghezza focale di una lente
	\section{Premesse teoriche}
	Una lente è un dispositivo ottico in grado di far concentrare o disperdere i raggi luminosi. \\
	\noindent Una lente divergente è un tipo di lente che, per l'appunto, fa divergere i raggi luminosi che la attraversano, facendo in modo che questi provengano da un punto dietro la lente stessa: siccome forma esclusivamente \emph{immagini virtuali} necessitiamo di una lente convergente con un potere diottrico maggiore in modulo rispetto a quello delle lente divergente. \\
	Per calcolare la lunghezza focale della lente divergente si utilizza la legge dei punti coniugati (nel caso di lenti sottili), che risulta essere
	\begin{equation}
		\frac{1}{p} + \frac{1}{q} = \frac{1}{f}
		\label{lens_maker}	
	\end{equation}
	dove $p$ e $q$ rappresentano, rispettivamente, la distanza fra la lente e lo schermo (quando essa non è a fuoco) e la distanza fra la lente e lo schermo con l'immagine a fuoco. \\
	Un fatto interessante è che la~(\ref{lens_maker}) mostra come per determinare la lunghezza focale $f$ sia necessario conoscere sia $p$ che $q$, ma se
	\begin{equation}
		\frac{1}{\hat{p}} \ll \frac{\sigma_q}{q^2} \, \implies \hat{p} \gg \frac{\hat{q}^2}{\sigma_q^2}
	\end{equation}
	allora si potrebbe trascurare il termine $\frac{1}{p}$ e misurare solamente $\frac{1}{q}$ (effettuando quindi una sola misura).
	\section{Strumenti e materiali}
	\textbf{Strumenti}	
	\begin{itemize}
		\item metro a nastro, con risoluzione $\pm 0.1 \si{\centi\meter}$
	\end{itemize}
	\textbf{Materiali}
	\begin{itemize}
		\item banco ottico con sorgente luminosa
		\item lente di lunghezza focale ignota
		\item schermo
		\item cassetta di lenti convergenti o divergenti
	\end{itemize}
	\section{Descrizione delle misure}
	Per semplificare la presa dei dati, la sorgente era coperta da un pezzo di plastica da cui era stato rimosso un triangolino: in questa maniera era più facile individuare se l'immagine era messa a fuoco siccome i bordi del triangolo si delineavano ancora più. \\
	Inizialmente ho verificato che la lente dal potere diottrico ignoto fosse divergente: questo è stato possibile prendendo la lente e ponendola davanti alla sorgente, osservando che non esisteva alcun punto in cui i raggi convergessero. (da completare la seconda parte) \\
	Come spiegato nelle premesse teoriche, per misurare la lunghezza focale necessitiamo di una lente convergente: per questo ho considerato una lente con potere diottrico di $+xx$ e abbiamo spostato lo schermo per mettere a fuoco l'immagine sullo schermo. A quel punto abbiamo posto la lente divergente fra lo schermo e la lente convergente e abbiamo misurato la distanza arbitraria $p$ fra lo schermo e la lente. Successivamente abbiamo spostato lo schermo in modo da rimettere a fuoco l'immagine e abbiamo misurato la distanza $q$ fra lo schermo e la lente. Iterando questo procedimento, abbiamo effettuato in totale $x$ misurazioni. \\
	Come incertezza non abbiamo utilizzato la risoluzione dello strumento siccome non conoscevamo bene la posizione del centro della lente e non si riusciva a trovare il punto esatto di messa a fuoco, ma solamente un intervallo in cui l'immagine sullo schermo risultava essere a fuoco; dunque abbiamo deciso di utilizzare un'incertezza di misura pari a $0.5 \, \si{\centi\meter}$
	\section{Analisi dei dati}
	I dati non soddisfano la condizione $\frac{1}{\hat{p}} \gg \frac{\sigma_q}{q^2}$ dunque non possiamo trascurare il termine $\frac{1}{p}$. Linearizziamo la ~(\ref{lens_maker}) ponendo $x = \frac{1}{q}$ e $y=\frac{1}{p}$ e si ottiene
	$$
		y = mx + \frac{1}{f}
	$$
	e possiamo dunque effettuare un fit utilizzando come modello teorico una semplice retta che ha $m$ come coefficiente angolare e $\frac{1}{f}$ come intercetta con l'asse y. Tuttavia si osserva che gli errori sulle $y$ non sono trascurabili e quindi abbiamo utilizzato gli errori efficaci. \\
	Riporto il grafico del fit \\
	
	Per valutare l'accordo fra il modello e i dati abbiamo calcolato il $\chi^2$ utilizzando sempre gli errori efficaci:
	\begin{equation}
		\chi^2 = \sum_{i=1}^n \frac{(y_i - f(x_i; \hat{\theta}_1, \ldots, \hat{\theta}_n))^2}{\sigma_{y_i}^2 + (\frac{df}{dx}(x_i; \hat{\theta}_1, \ldots, \hat{\theta}_n))^2 \sigma_{x_i}^2}
	\end{equation}
\end{document}